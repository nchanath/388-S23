\documentclass{exam}

\usepackage{colortbl}
\usepackage{fancyvrb}
\usepackage{listings} 
\usepackage{amsmath}

\setlength{\textwidth}{6.5in}
\setlength{\textheight}{9in}
\setlength{\oddsidemargin}{0in}
\setlength{\evensidemargin}{0in}
\setlength{\topmargin}{0in}
\setlength{\headheight}{0in}
\setlength{\headsep}{0in}
\setlength{\footskip}{0.5in}

\def\COURSENAME{Cryptography}
\def\COURSECODE{CSCI/MATH 388}
\def\YEAR{2023}

\definecolor{lightgreen}{cmyk}{.25,0,.25,0}
\definecolor{black}{rgb}{0,0,0}
\definecolor{dgray}{rgb}{0.1,0.1,0.1}
\definecolor{lgray}{rgb}{0.5,0.5,0.5}
\definecolor{dblue}{rgb}{.2,.4,.6}
\definecolor{lblue}{rgb}{.2,0,.7}
\definecolor{llblue}{rgb}{.2,0,.6}
\definecolor{gold}{rgb}{0.8512,0.6042,0.0000}
\definecolor{dred}{rgb}{0.9,0,0}
\definecolor{lred}{rgb}{1.0000,0.2013,0.1405}
\definecolor{orange}{rgb}{1,0.3,0}
\definecolor{dgreen}{rgb}{.01,.6,.06}
\definecolor{lgreen}{rgb}{0.2610,0.8730,0.2875}
\definecolor{peachpuff}{rgb}{1.0000,0.7197,0.5181}
\definecolor{cornflowerblue}{rgb}{0.0859,0.2912,0.8708}
\definecolor{deepskyblue}{rgb}{0.0000,0.5227,1.0000}
\definecolor{lightskyblue}{rgb}{0.1946,0.6324,1.0000}
\definecolor{darkturquoise}{rgb}{0.0000,0.6324,0.6861}
\definecolor{darkseagreen}{rgb}{0.2264,0.5227,0.2498}
\definecolor{lemon}{rgb}{0.8512,0.8416,0.5181}
\definecolor{dred}{rgb}{0.9,0,0}


\newcommand{\T}{\mathbf{T}}
\newcommand{\F}{\mathbf{F}}
\newcommand{\Q}{\mathbf{Q}}


\newcommand{\bheading}[1]{\vspace{10pt} \noindent \textbf{#1}}

\newenvironment{problemstmt}[1]{\begin{rm}
  \vspace{.3in}
  \noindent \textbf{Problem #1.}}
  {\end{rm}
  \vspace{.3in}
  \hrule}

\def\nextt{\:;\:}
\def\gets{\;\leftarrow\;}
\def\getsr{\stackrel{{\scriptscriptstyle \$}}{\leftarrow}}
\newcommand{\Colon}{\;:\;}
\newcommand{\Prob}[1]{\Pr\left[\: #1 \:\right]}
\newcommand{\ProbExp}[2]{\Pr\left[\: #1 \: : \: #2\: \right]}
\newcommand{\Probb}[2]{{\Pr}_{#1}\left[\: #2 \:\right]}
\newcommand{\Probc}[2]{\Pr\left[\: #1 \:{\left|\right.}\:#2\:\right]}
\newcommand{\Probcc}[3]{{\Pr}_{#1}\left[\: #2 \:\left|\right.\:#3\:\right]}
\newcommand{\CondProb}[2]{{\Pr}\left[\: #1\:\left|\right.\:#2\:\right]}
\newcommand{\suchthat}{{\mbox{s.t.\ }}}
\newcommand{\Exp}{\mathbf{Exp}}
\newcommand{\Adv}{\mathbf{Adv}}
\newcommand{\advA}{A}
\newcommand{\ExpH}[3]{\Exp^{\mathrm{cr#1}\mbox{-}\mathrm{kk}}_{#2}(#3)}
\newcommand{\AdvH}[3]{\Adv^{\mathrm{cr#1}\mbox{-}\mathrm{kk}}_{#2}(#3)}
\newcommand{\AdvHz}[2]{\Adv^{\mathrm{cr0}}_{#1}(#2)}
\newcommand{\DS}{\mathcal{DS}}
\newcommand{\MA}{\mathcal{MA}}
\newcommand{\Enc}{\mathcal{E}}
\newcommand{\Dec}{\mathcal{D}}
\newcommand{\Sign}{\mathrm{Sign}}
\newcommand{\KG}{\mathsf{KG}}
\newcommand{\mac}{\mathcal{MAC}}
\newcommand{\verify}{\mathcal{VF}}
\newcommand{\ctr}{ctr}
\newcommand{\xor}{\oplus}
\newcommand{\bits}{\{0,1\}}
\newcommand{\SE}{\mathcal{SE}}
\newcommand{\K}{\mathcal{K}}
\newcommand{\E}{\mathcal{E}}
\newcommand{\D}{\mathcal{D}}
\newcommand{\MAC}{\mathsf{Tag}}
\newcommand{\VF}{\mathsf{VF}}
\newcommand{\lxlcpa}{\mathrm{lxl}\mbox{-}\mathrm{cpa}}
\newcommand{\pad}{\mathsf{pad}}
\newcommand{\IV}{\mathsf{IV}}

\newcommand{\op}{op}
\newcommand{\Z}{\mathbf{Z}}
\newcommand{\R}{\mathbf{R}}
\newcommand{\N}{\mathbf{N}}
\newcommand{\true}{\mathbf{T}}
\newcommand{\false}{\mathbf{F}}
\newcommand{\ConsE}{\mathsf{Cons}_E}

\newcommand{\hash}{\mathsf{H}}
\newcommand{\tostr}[1]{\langle #1 \rangle}
%\newcommand{\tonum}[1]{\llbracket #1 \rrbracket}
\newcommand{\tonum}[1]{[\![ #1 ]\!]}

\newcommand{\INDCPA}{\mathsf{IND}\mbox{-}\mathsf{CPA}}
\newcommand{\INDCCA}{\mathsf{IND}\mbox{-}\mathsf{CCA}}
\newcommand{\WUFCMA}{\mathsf{WUF}\mbox{-}\mathsf{CMA}}
\newcommand{\SUFCMA}{\mathsf{SUF}\mbox{-}\mathsf{CMA}}
\newcommand{\PRF}{\mathsf{PRF}}
\newcommand{\PRPCPA}{\mathsf{PRP}\mbox{-}\mathsf{CPA}}
\newcommand{\PRPCCA}{\mathsf{PRP}\mbox{-}\mathsf{CCA}}
\newcommand{\Coll}{\mathsf{Coll}}
\newcommand{\padwithzeros}{\mathsf{padwithzeros}}

\newcommand{\subroutinefont}[1]{\texttt{#1}}
\newcommand{\ExpHcoll}[2]{\Exp^{\mathrm{coll}}_{#1}(#2)}
\newcommand{\AdvHcoll}[2]{\Adv^{\mathrm{coll}}_{#1}(#2)}
\newcommand{\Initialize}{\subroutinefont{Initialize}}
\newcommand{\Finalize}{\subroutinefont{Finalize}}

\newcommand{\DES}{\mathsf{DES}}
\newcommand{\AES}{\mathsf{AES}}
\newcommand{\DESX}{\mathsf{DESX}}
\newcommand{\ECB}{\mathsf{ECB}}
\newcommand{\CBC}{\mathsf{CBC}}
\newcommand{\CTR}{\mathsf{CTR}}
\newcommand{\CTRC}{\mathsf{CTRC}}
\newcommand{\CTRCEnc}{\mathsf{CTRC}\mbox{-}\mathsf{Enc}}
\newcommand{\CTRCDec}{\mathsf{CTRC}\mbox{-}\mathsf{Dec}}
\newcommand{\PRFTag}{\mathsf{PRF}\mbox{-}\mathsf{Tag}}


\newcommand{\pk}{\mathsf{pk}}
\newcommand{\sk}{\mathsf{sk}}

\newcommand{\cbcr}{CBC\$}
\newcommand{\cbcc}{CBCC}
\newcommand{\cfbr}{CFB\$}
\newcommand{\cfbc}{CFBC}
\newcommand{\ofbr}{OFB\$}
\newcommand{\ofbc}{OFBC}
\newcommand{\ctrr}{CTR\$}
\newcommand{\ctrc}{CTRC}
\newcommand{\prf}{\mathrm{prf}}
\newcommand{\prpcpa}{\mathrm{prp\mbox{-}cpa}}
\newcommand{\prpcca}{\mathrm{prp\mbox{-}cca}}
\newcommand{\indcpa}[1]{\mathrm{ind\mbox{-}cpa\mbox{-}#1}}
\newcommand{\indcpaa}{\mathrm{ind\mbox{-}cpa}}
\newcommand{\indcca}[1]{\mathrm{ind\mbox{-}cca\mbox{-}#1}}
\newcommand{\indccaa}{\mathrm{ind\mbox{-}cca}}
\newcommand{\prcpa}{\mathrm{pr\mbox{-}cpa}}
\newcommand{\algfont}[1]{\mathsf{#1}}



\newcommand{\boldit}[1]{\textbf{\color{lblue} #1}}
\newcommand{\term}[1]{\boldit{#1}}

\newcommand{\verylongleftarrow}[1]
      {\setlength{\unitlength}{.01in}
      \begin{picture}(#1,1) \put(#1,0){\vector(-1,0){#1}} \end{picture}}
\newcommand{\verylongrightarrow}[1]             %longleft and rightgoing arrows
      {\setlength{\unitlength}{.01in}           %for protocols
      \begin{picture}(#1,1) \put(0,0){\vector(1,0){#1}} \end{picture}}
\newcommand{\leftgoing}[2]{{\stackrel{{\displaystyle #2}} {\verylongleftarrow{#1}}}}
\newcommand{\rightgoing}[2]{{\stackrel{{\displaystyle #2}} {\verylongrightarrow{#1}}}}

\newcommand{\leftgoinga}[1]{\leftgoing{230}{#1} }
\newcommand{\rightgoinga}[1]{\rightgoing{230}{#1} }

\newcommand{\leftgoingb}[1]{\leftgoing{300}{#1} }
\newcommand{\rightgoingb}[1]{\rightgoing{300}{#1} }

\newcommand{\Dm}{\mathbf{\cal D}}
\newcommand{\Rg}{\mathbf{\cal R}}
\newcommand{\ufcma}{\mathrm{uf\mbox{-}cma}}
\newcommand{\shf}{\algfont{shf}}
%\newcommand{\shfone}{\phantom{\shf1_K}}
\newcommand{\shapad}{\mathsf{shapad}}
\newcommand{\shaone}{\mathsf{SHA1}}
\newcommand{\shfone}{\mathsf{SHF1}}

\newcommand{\orfont}[1]{\mathtt{#1}}
\newcommand{\schemefont}[1]{\mathsf{#1}}
\newcommand{\flagfont}[1]{\mathsf{#1}}


\newcommand{\InitializeCode}{\underline{\textbf{proc} \procfont{Initialize}}}
\newcommand{\FinalizeCode}[1]{\underline{\textbf{proc} \procfont{Finalize}(#1)}}
\newcommand{\returnst}{\Rightarrow \true}
\newcommand{\returnsf}{\Rightarrow \false}
\newcommand{\g}{{g}}
\newcommand{\simEnc}{{SimEnc}}
\newcommand{\greal}{\g \text{ \texttt{real}}}
\newcommand{\grand}{\g \text{ \texttt{rand}}}
\newcommand{\EncO}{\orfont{Enc}}
\newcommand{\DecO}{\orfont{Dec}}
\newcommand{\DecVO}{\VfO}
\newcommand{\Tag}{\algfont{Tag}}
\newcommand{\Vf}{\algfont{Vf}}
\newcommand{\TagO}{\orfont{Tag}}
\newcommand{\VfO}{\orfont{Vf}}
\newcommand{\ExpPRF}[2]{\Exp^{\mathrm{prf}}_{#1}(#2)}
\newcommand{\AdvPRF}[2]{\Adv^{\mathrm{prf}}_{#1}(#2)}
\newcommand{\ExpSEIndCPA}[2]{\Exp^{\mathrm{ind\mbox{-}cpa}}_{#1}(#2)}
\newcommand{\ExpSEIndCCA}[2]{\Exp^{\mathrm{ind\mbox{-}cca}}_{#1}(#2)}
\newcommand{\AdvSEIndCPA}[2]{\Adv^{\mathrm{ind\mbox{-}cpa}}_{#1}(#2)}
\newcommand{\AdvSEIndCCA}[2]{\Adv^{\mathrm{ind\mbox{-}cca}}_{#1}(#2)}
\newcommand{\ExpSEIndCPAb}[3]{\Exp^{\mathrm{ind\mbox{-}cpa\mbox{-}#3}}_{#1}(#2)}
\newcommand{\ExpSEIndCCAb}[3]{\Exp^{\mathrm{ind\mbox{-}cca\mbox{-}#3}}_{#1}(#2)}
\newcommand{\AdvSEIndCPAb}[2]{\Adv^{\mathrm{ind\mbox{-}cpa*}}_{#1}(#2)}
\newcommand{\AdvSEIndCCAb}[2]{\Adv^{\mathrm{ind\mbox{-}cca*}}_{#1}(#2)}
\newcommand{\ExpSEIndCPAr}[3]{\Exp^{\mathrm{ror\mbox{-}ind\mbox{-}cpa\mbox{-}#3}}_{#1}(#2)}
\newcommand{\ExpSEIndCCAr}[3]{\Exp^{\mathrm{ror\mbox{-}ind\mbox{-}cca\mbox{-}#3}}_{#1}(#2)}
\newcommand{\AdvSEIndCPAr}[2]{\Adv^{\mathrm{ror\mbox{-}ind\mbox{-}cpa}}_{#1}(#2)}
\newcommand{\AdvSEIndCCAr}[2]{\Adv^{\mathrm{ror\mbox{-}ind\mbox{-}cca}}_{#1}(#2)}
\newcommand{\ExpMAW}[2]{\Exp^{\mathrm{wuf}\mbox{-}\mathrm{cma}}_{#1}(#2)}
\newcommand{\AdvMAW}[2]{\Adv^{\mathrm{wuf}\mbox{-}\mathrm{cma}}_{#1}(#2)}
\newcommand{\ExpMAS}[2]{\Exp^{\mathrm{suf}\mbox{-}\mathrm{cma}}_{#1}(#2)}
\newcommand{\AdvMAS}[2]{\Adv^{\mathrm{suf}\mbox{-}\mathrm{cma}}_{#1}(#2)}
\newcommand{\AdvPRG}[2]{\Adv^{\mathrm{prg}}_{#1}(#2)}
\newcommand{\AdvPRGb}[2]{\Adv^{\mathrm{prg*}}_{#1}(#2)}
\newcommand{\ExpPRG}[2]{\Exp^{\mathrm{prg}}_{#1}(#2)}
\newcommand{\ExpPRGa}[1]{\Exp^{\mathrm{prg}}_{#1}}
\newcommand{\ExpPRGb}[3]{\Exp^{\mathrm{prg}\mbox{-}#3}_{#1}(#2)}


\newcommand{\prgG}{G}
\newcommand{\Succ}{\mathbf{Succ}}
\newcommand{\AEscheme}{\schemefont{AE}}
\newcommand{\SEscheme}{\schemefont{SE}}
\newcommand{\MAscheme}{\schemefont{MA}}
\newcommand{\PKEscheme}{\schemefont{PKE}}

\newcommand{\hashfn}{H}
\newcommand{\advB}{B}
\newcommand{\AdvHpre}[2]{\Adv^{\mathrm{pre}}_{#1}(#2)}
\newcommand{\AdvHsec}[2]{\Adv^{\mathrm{sec}}_{#1}(#2)}
\newcommand{\ExpHpre}[2]{\Exp^{\mathrm{pre}}_{#1}(#2)}
\newcommand{\ExpHsec}[2]{\Exp^{\mathrm{sec}}_{#1}(#2)}
\renewcommand{\eqref}[1]{Equation~(\ref{#1})}

\newcommand{\bad}{\mathsf{bad}}


\def\getsr{\stackrel{{\scriptscriptstyle \$}}{\leftarrow}}


\newcommand{\tableminipage}{\begin{minipage}{0.5in}
\begin{tabular}{|l|l|} \hline 
  \textbf{output} & \hspace{4.5in} \\ \hline
\end{tabular} 
\end{minipage}}




\begin{document}


\begin{tabbing}
  \`\=\kill
  \textbf{\COURSECODE:} \COURSENAME \` Spring \YEAR \\
  Department of Computer Science \` Reed College \\
  \textbf{Problem Set 3:}\` \textbf{Instructor:} Chanathip Namprempre
\end{tabbing}


\hrule

\vspace{.4in}

\begin{center}
\textbf{\Large Problem Set 3}
\end{center}


\vspace{.2in}


\begin{questions}

  \question 
  Consider the following security definition for pseudorandom generator.

  \bigskip
  \fbox{%
    \parbox{0.9\linewidth}{%
      Let $m$ and $n$ be positive integers. Let $\prgG:\bits^m \rightarrow \bits^n$ be a pseudorandom generator, and
      let $\advA$ be an adversary against $\prgG$. We define the following
      subroutines, experiment, and advantage function.
      
      \bigskip
      \begin{tabular}{l|l}
      \begin{minipage}{\textwidth}
        \begin{tabbing}
          123\=123\=123\=\kill
          Subroutine $\Initialize(w)$ \\
          \> If $w = 0$ \\
          \> \> then $y \getsr \bits^n$ \\
          \> \> else $s \getsr \bits^m \;;\; y \gets \prgG(s)$ \\
          \> Return $y$
        \end{tabbing}
      \end{minipage}
      &
      \begin{minipage}{\textwidth}
        \begin{tabbing}
          123\=123\=\kill
          Experiment $\ExpPRGb{\prgG}{\advA}{w}$ \\
          \> $y \getsr \Initialize(w)$ \\
          \> $d \getsr \advA(y)$ \\
          \> Return $d$ \\
        \end{tabbing}
      \end{minipage}
      \end{tabular}

      \bigskip
      \noindent We define the \textit{prg* advantage} of an
      adversary $\advA$ attacking $\prgG$ as
      \[
      \AdvPRGb{\prgG}{\advA} = \Prob{\ExpPRGb{\prgG}{\advA}{1} \Rightarrow 1} -  \Prob{\ExpPRGb{\prgG}{\advA}{0} \Rightarrow 1}  \;.
      \]
    }
  }

  Recall the definition of $\Adv^{\mathrm{prg}}$ defined in the textbook and studied in class. Prove that, for all $\prgG$ and $\advA$, 
  \[ \AdvPRGb{\prgG}{\advA} = \AdvPRG{\prgG}{\advA}\;. \]


  \question Let $m$ and $n$ be positive integers, and let $G_1:\bits^m \rightarrow \bits^n$ and $G_2:\bits^m \rightarrow \bits^n$ be pseudorandom generators. Define a pseudorandom generator $G: \bits^m \rightarrow \bits^{2n}$ as follows. For any $s \in \bits^m$,
  \[ G(s) \;=\; G_1(s) \| G_2(s)\;. \]
  Suppose that $G_1$ and $G_2$ are secure under the PRG security notion. Is $G$ necessarily a secure PRG? Prove your answer. Be sure to provide a complete proof. Specifically, if you answer yes, specify a reduction along with an analysis relating the advantages of relevant adversaries and their resource usage. If you answer no, specify a counterexample, an attack, and an analysis of the adversary's advantage and resource usage. As always, an adversary requiring a minimal amount of resources while achieving a high advantage value is better.

  \question Let $m$ and $n$ be positive integers, and let $G_1:\bits^m \rightarrow \bits^n$ and $G_2:\bits^m \rightarrow \bits^n$ be pseudorandom generators. Define a pseudorandom generator $G: \bits^{2m} \rightarrow \bits^{2n}$ as follows. For any $s_1,s_2 \in \bits^m$,
  \[ G(s_1s_2) \;=\; G_1(s_1) \| G_2(s_2)\;. \]
  Suppose that $G_1$ and $G_2$ are secure under the PRG security notion. Is $G$ necessarily a secure PRG? Prove your answer. Be sure to provide a complete proof. Specifically, if you answer yes, specify a reduction along with an analysis relating the advantages of relevant adversaries and their resource usage. If you answer no, specify a counterexample, an attack, and an analysis of the adversary's advantage and resource usage. As always, an adversary requiring a minimal amount of resources while achieving a high advantage value is better.


  \question Let $n$ be a positive integer. Recall that $[\KG]$ denotes the set of all possible keys output by the algorithm $\KG$. Let $\MAscheme = (\KG,\Tag,\Vf)$ be a MAC scheme secure under the SUF-CMA security notion, and let $\bits^n$ be the message space for $\MAscheme$. We define $\MAscheme' = (\KG,\Tag',\Vf')$ where, for all $M \in \bits^{2n}$, for all $K \in [\KG]$,
  \[ \Tag'_K(M)  \;=\;  \Tag_K( M[1]) \| \Tag_K( M[2] )  \]
  where $M = M[1]M[2]$ and $|M[1]| = |M[2]|$.
  \begin{parts}
    \part Write a deterministic and stateless algorithm $\Vf'$ that would ensure that $\MAscheme'$ satisfies the correctness condition. 

    \part Is $\MAscheme'$ necessarily a secure MAC scheme? Prove your answer. Be sure to provide a complete proof. Specifically, if you answer yes, specify a reduction along with an analysis relating the advantages of relevant adversaries and their resource usage. If you answer no, specify a counterexample, an attack, and an analysis of the adversary's advantage and resource usage. As always, an adversary requiring a minimal amount of resources while achieving a high advantage value is better.
  \end{parts}

  \question Let $n$ be a positive integer. Recall that $[\KG]$ denotes the set of all possible keys output by the algorithm $\KG$. Let $\MAscheme = (\KG,\Tag,\Vf)$ be a MAC scheme secure under the SUF-CMA security notion, and let $\bits^n$ be the message space for $\MAscheme$. We define $\MAscheme' = (\KG,\Tag',\Vf')$ where, for all $M \in \bits^{n}$, for all $K \in [\KG]$,
  \[ \Tag'_K(M)  \;=\;  \Tag_K(M) \| \Tag_K(M)  \;. \]
  \begin{parts}
    \part Write a deterministic and stateless algorithm $\Vf'$ that would ensure that $\MAscheme'$ satisfies the correctness condition. 

    \part Is $\MAscheme'$ necessarily a secure MAC scheme? Prove your answer. Be sure to provide a complete proof. Specifically, if you answer yes, specify a reduction along with an analysis relating the advantages of relevant adversaries and their resource usage. If you answer no, specify a counterexample, an attack, and an analysis of the adversary's advantage and resource usage. As always, an adversary requiring a minimal amount of resources while achieving a high advantage value is better.
  \end{parts}

  \question Let $n$ be a positive integer. Recall that $[\KG]$ denotes the set of all possible keys output by the algorithm $\KG$. Let $\MAscheme_1 = (\KG,\Tag_1,\Vf_1)$ and $\MAscheme_2 = (\KG,\Tag_2,\Vf_2)$ be MAC schemes secure under the SUF-CMA security notion, and let $\bits^n$ be the message space for $\MAscheme_1$ and $\MAscheme_2$. We define $\MAscheme_3 = (\KG,\Tag_3,\Vf_3)$ where, for all $M \in \bits^{n}$, for all $K \in [\KG]$,
  \[ \Tag_3(K, M)  \;=\;  \Tag_1( K, M) \| \Tag_2( K, M )  \;. \]
  (Note that the notation here is slightly different from the previous question to avoid potential confusion regarding the algorithm name and the subscript $K$.)
  \begin{parts}
    \part Write a deterministic and stateless algorithm $\Vf_3$ that would ensure that $\MAscheme_3$ satisfies the correctness condition. 

    \part Is $\MAscheme_3$ necessarily a secure MAC scheme? Prove your answer. Be sure to provide a complete proof. Specifically, if you answer yes, specify a reduction along with an analysis relating the advantages of relevant adversaries and their resource usage. If you answer no, specify a counterexample, an attack, and an analysis of the adversary's advantage and resource usage. As always, an adversary requiring a minimal amount of resources while achieving a high advantage value is better.
  \end{parts}
  
\end{questions}
\end{document}
